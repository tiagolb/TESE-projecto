%!TEX root = ../report.tex

% 
% Conclusions
% 

\section{Conclusions}
\label{sec:conclusion}

% REVER
This project aims at developing a new security solution for critical mobile applications named \emph{ViewZone}. With this solution, applications are able to store and display sensitive data to the user. This is achieved by using ARM's TrustZone technology, which allows for the execution of isolated, yet integrated, environments in TrustZone-enabled mobile devices. This isolation mitigates information leakage problems inherent to shared resource architectures. Using \emph{ViewZone}, client Android applications can be integrated with trusted services running in the isolated secure domain in order to securely display and store sensitive information. These trusted services are able to securely control the display peripherals and drivers by implementing a self-contained secure display controller and a secure isolated framebuffer. Moreover, unlike previous systems, \emph{ViewZone} mitigates denial-of-service attacks by relying on secure hardware interrupts for a reliable world-switch. In a near future, and in order to evaluate this new security framework, an instance of \emph{ViewZone} and of a secure personal health record mobile application will be implemented. This implementation will allow to assess both \emph{ViewZone}'s performance in real life scenarios as well as its usefulness in the development of critical applications.