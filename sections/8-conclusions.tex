%!TEX root = ../report.tex

% 
% Conclusions
% 

\section{Conclusions}
\label{sec:conclusion}

% REVER
This project aims at developing a new security system for mobile applications named \emph{ViewZone}. With this solution, apps can display sensitive data to the user without exposing such information to the OS. This is achieved by using ARM's TrustZone technology, which allows for the execution of isolated environments in TrustZone-enabled mobile devices, thus mitigating information leakage problems inherent to shared resources. Using \emph{ViewZone}, Android applications can use \emph{ViewZone}'s trusted service running in the isolated secure domain in order to securely display sensitive information. This trusted service can securely control the display peripherals and drivers by implementing a self-contained secure display controller and a secure isolated framebuffer. Moreover, unlike previous systems, \emph{ViewZone} mitigates denial-of-service attacks by relying on secure hardware interrupts for a reliable world-switch. In a near future, and in order to evaluate this new security framework, an instance of \emph{ViewZone} and of a secure personal health record mobile application will be implemented. This implementation will allow to assess both \emph{ViewZone}'s performance in real life scenarios as well as its usefulness in the development of critical applications.