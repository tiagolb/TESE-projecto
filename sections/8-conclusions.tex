%!TEX root = ../report.tex

% 
% Conclusions
% 

\section{Conclusions}
\label{sec:conclusion}

% REVER
This project aims to develop a new security framework for critical mobile applications named \emph{ViewZone}. With this framework, security sensitive applications are able to display, modify and store sensitive data to the user, through the execution of trusted services. This is achieved by using ARM's TrustZone technology, which allows for the execution of isolated, yet integrated, environments in TrustZone-enabled mobile devices. This isolation mitigates information leakage problems inherent to shared resource architectures. Using \emph{ViewZone} client Android applications can be integrated with trusted services running in the isolated secure domain in order to securely display, edit and store secure sensitive information. These trusted services are able to securely control the display and input peripherals and drivers by implementing a self-contained secure display controller, a secure isolated framebuffer and a secure input driver, for instance for a touchscreen device. Moreover, unlike previous systems, \emph{ViewZone} mitigates denial-of-service attacks by relying on secure hardware interrupts for a reliable world-switch. In a near future, and in order to evaluate this new security framework, an instance of \emph{ViewZone} and of a secure personal health record mobile application will be implemented. This implementation will allow to assess both \emph{ViewZone} performance in real life scenarios as well as its usefulness in the
development of critical application which must rely on trusted user interfaces to interact with the user.