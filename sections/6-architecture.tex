%!TEX root = ../report.tex

% 
% Architecture
% 
% (2/3pgs)
\section{Architecture}
\label{sec:architecture}

Your proposed architecture. Can have lots of pictures and bullet points so it is easy to understand.

%DIZER QUE FOI INSPIRADA NO TRABALHO DE DROIDVAULT PARA O STORAGE DE FILES E EXPLICAR COMO SE PROCEDE

%INSPIRADA NO TRABALHO DE TRUSTOTP PARA A TRASIÇAO ENTRE OS MUNDOS DE MANEIRA A CONSEGUIR GARANTIR AO UTILIZADOR QUE A INFORMAÇAO QUE ESTÁ A VER É CORRECTA E EXPLICAR COMO SE PROCEDE

%TRUSTED UI BASEADO NO FACTO DE USAR UMA FRAMEWORK QUE SUPORTE ISTO, OU SEJA QUE CONTENHA NA SUA TCB AS DRIVES NECESSARIAS PARA QUE ISTO SEJA POSSIVEL? OU DIZER QUE SE ESCOLHE UMA ESTRATEGIA À TRUSTUI PARA USAR AS DRIVERS JÁ PRESENTES NO RICH OS? E EXPLICAR COMO SE PROCEDE

%MOSTRAR UM DESENHO A MOSTRAR COMO O FICHEIRO FICA GRAVADO NO LADO UNTRUSTED MAS É TRUSTED PORQUE O MUNDO SEGURO É QUE PROCEDE À CIFRA E DECIFRA. A CHAVE É A UNICA COISA QUE FICA GRAVADA NO MUNDO SEGURO

%FALAR DA APP DE PROTOTIPO, UM FICHEIRO JÁ CIFRADO COM UM PAR DE CHAVES É SACADO USANDO O UNTRUSTED OS. O USER ACEDE À APP ANDROID QUE FAZ A PONTE ENTRE OS DOIS MUNDOS E MANDA LER AQUELE FICHEIRO, O MUNDO SEGURO TOMA CONTA DA MEMORIA E DISPLAY E VAI BUSCAR O FICHEIRO INDICADO. JÁ NO MUNDO SEGURO DECIFRA O FICHEIRO E MOSTRA-O AO UTILIZADOR. O UTILIZADOR PODE EDITAR O FICHEIRO E MANDAR GUARDÁ-LO E DÁ-SE ORIGEM AO PRCESSO INVERSO, EM QUE O MUNDO SEGURO PEGA NO FICHEIRO E CIFRA-O, GUARDA-O NA ZONA DE MEMORIA E RETORNA AO MUNDO INSEGURO QUE PROCEDE A FAZER A STORAGE DESSE FICHEIRO.