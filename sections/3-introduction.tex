%!TEX root = ../report.tex

% 
% Introduction
% 
% (2/3pgs)
\section{Introduction}

%General description of the problem and its context, current solutions, and road map of the project.
%1. Contexto, importância da privacidade (nomeadamente no âmbito da e-Health)
%2. Motivação, explicação dos problemas/desafio técnicos em tornar as apps seguras (foco em e-Health)
%3. Definir objectivo e requisitos (dependem dos problemas técnicos)
%4. Apresentar a arquitectura da solução

% ** MOTIVATION AND CONTEXT **

Mobile devices are becoming the predominant platform for simple everyday computing activities.  A recent study~\cite{comscore} from early 2015, with key statistics for the U.S. market, shows that smartphones and tablets dominate digital media time over the \ac{PC}, i.e., Internet searches, games and other digital content consumption, with the trend being to continue raising. This shift is explained by the fact that actions that required powerful desktop computers can now be easily performed on mobile devices.

Along with the proliferation of mobile devices, mobile applications (apps) have started to handle privacy and security sensitive data, such as photos, health records, banking information, location data and general documentation. Mobile health (mHealth) applications, for example, handle medical data, which, according to Reuters~\footnote{http://www.reuters.com/article/2014/09/24/us-cybersecurity-hospitals-idUSKCN0HJ21I20140924}, is more valuable than credit card information. In 2013, Research2Guidance~\cite{research2guidance} reported the existence of more than 97.000 mHealth apps across 62 app stores, with the top 10 apps generating up to 4 million free and 300.000 paid downloads per day. This market is expected to grow even further, from a \$6.21 billion revenue in 2013 to \$23.49 billion by 2018~\cite{marketsandmarkets}.

However, as mobile devices store and process increasing amounts of security sensitive data, specially from banking and health monitoring applications, they become more attractive targets for data stealing malware. Data becomes particularly vulnerable when encryption cannot be used, namely when it must be displayed on the device's screen. For example, when browsing through sensitive health records, the data must be written in the framebuffer in its unencrypted raw form so that it is readable by the user. Thus, in a malware compromised operating system, data can easily fall prey to an attacker who can steal it from the framebuffer and leak it out.

% JA ESTA NA 2.1
%To prevent negligent development of health care systems, and protect sensitive data from being disclosed, regulatory laws such as the \ac{HIPAA} have been established. These laws comprise the standard for electronic health care transactions and must be followed by all developers when managing sensitive health data. But regulatory laws are not enough to protect sensitive data, and criminals are motivated by the valuable health care information to perform critical attacks on hospital networks around the world, consequently leaking or stealing health records of millions of patients. In September 2014, a group of hackers attacked UCLA's Hospital network, accessing computers with sensitive records of 4.5 million people. According to \ac{CNN}~\footnote{http://money.cnn.com/2015/07/17/technology/ucla-health-hack/}, among the stolen records were the names, medical information, Social Security numbers, Medicare numbers, health plan IDs, birthdays and physical addresses of UCLA's patients.
% ** PREVIOUS SOLUTIONS AND TECHNICAL DIFFICULTIES **
%In the mobile context, data is even more exposed and vulnerable due to the inherent portability of these devices, the sharing of information to third-party advertisers by device manufacturers or mobile app developers, unregulated management of sensitive medical information, specially because regulatory laws such as \ac{HIPAA} do not account for the mobile sector, and because of security flaws on medical or consumer device software.

%REVER ESTA PARTE
Currently, providing secure output capability for commodity mobile platforms, such as Android or iOS, is very challenging. Essentially, to protect sensitive data, mobile app developers rely on mechanisms implemented by the operating system. Such mechanisms include, access control~\cite{smalley2013security,bugiel2011practical,nauman2010apex,kern2012permission,conti2011crepe,russello2012moses,heuser2014asm,backes2014android,wang2015deepdroid,drm}, application communication monitoring~\cite{ongtang2012semantically,dietz2011quire,bugiel2011xmandroid} and privacy enhancement systems~\cite{beresford2011mockdroid,zhou2011taming,shebaro2014identidroid,enck2014taintdroid}, either from native Android, iOS and Windows or from extensions. But these mechanisms do not offer support for trusted user interfaces, i.e., to securely input and display sensitive information, and rely on ad-hoc \ac{OS} and application-level methodologies, which in most cases depend upon a very complex \ac{TCB} code.

The research community has developed execution environments, based on microkernels and trusted hardware technologies, with the goal of reducing the \ac{TCB} and offering an isolated environment for security critical applications. But these systems generally do not support, or make it very hard for application developers to offer trusted user interfaces to the user.

\subsection{Goals}
% ** OBJECTIVES ** 
The goal of this project is to fill in the security gap in the mobile application market by proposing \emph{ViewZone}, a security solution for developing and executing critical mobile applications which require trusted display and input sensitive data in several formats, such as images and text. This ensures that data displayed by the device and input by the user are not intercepted by a malware compromised operating system. \emph{ViewZone} leverages ARM's TrustZone technology in order to maintain a small, dedicated \ac{TCB} and support an isolated execution environment from that of a full-featured operating system, such as Android.
%% RESTRICTIONS
To support the features referenced above, the final implementation must guarantee the following requirements:

\paragraph*{\textbf{Small Trusted Computing Base\\}} To ensure the correctness of \emph{ViewZone} and of the trusted services running on top of it, this implementation must have a small \ac{TCB}, comprising of a few KLOCs. This can be done by using TrustZone to isolate the secure service responsible for the trusted display and input from the rich operating system, and thus relying only on essential mechanism controlled by the secure hardware.

\paragraph*{\textbf{Support for General Applications\\}} The implementation should allow generic Android applications to securely display sensitive content within its context without relying on operating system mechanisms, which may be compromised. This means that the user must not be aware of a context change between the untrusted generic application and its trusted counterpart.

\paragraph*{\textbf{Developer Friendly\\}} Developers must be able to easily specify what needs to be displayed in the trusted environment and how the input should be handled, as well as how it affects the data being displayed.\\

%% CONTRIBUTIONS **
In summary this work expects to contribute with:
\begin{itemize}
	\item The design of a TrustZone-based software system that provides secure output channels for Android applications while depending on small TCB.	
	\item Implementation, on a development board, of a prototype of this new security framework.
	\item Implementation of a mobile health application using the prototyped framework.
	\item Experimental evaluation of the prototype and mobile health app.
\end{itemize}

%
%% ORGANIZATION **
The remainder of this document proceeds as follows. Section~\ref{sec:relatedWork} discusses the related work. Section~\ref{sec:architecture} introduces the architecture of \emph{ViewZone}. Section~\ref{sec:evaluation} highlights the evaluation methodology and implementation and is followed by Section~\ref{sec:workplan}, which describes the work plan for this project. Section~\ref{sec:conclusion} concludes this report.