%!TEX root = ../report.tex

% 
% Introduction
% 
% (2/3pgs)
\section{Introduction}

%General description of the problem and its context, current solutions, and road map of the project.
%1. Contexto, importância da privacidade (nomeadamente no âmbito da e-Health)
%2. Motivação, explicação dos problemas/desafio técnicos em tornar as apps seguras (foco em e-Health)
%3. Definir objectivo e requisitos (dependem dos problemas técnicos)
%4. Apresentar a arquitectura da solução

% ** MOTIVATION AND CONTEXT **

Mobile devices are becoming the predominant device for simple everyday computing activities. Actions previously performed by powerful desktop computers can now be easily replicated on mobile devices. A recent study \cite{comscore} from early 2015, with key statistics for the U.S. market, shows that mobile devices, such as smartphones and tablets, dominate digital media time over the \ac{PC}, i.e., Internet searches, games and other digital content consumption, with the trend being to continue raising. Due to the prolific use of mobile devices, mobile applications (apps) start to handle more privacy and security sensitive data. Most notably, these apps are handling photos, health and banking information, location and general documentation. However, as mobile devices store and process increasing amounts of security sensitive data, specially from banking and health monitoring applications, they become more attractive targets for data stealing malware.

Parallel to mobile app market's growth, the number of mobile health apps (mHealth) available is increasing rapidly. In 2013, Research2Guidance~\cite{research2guidance} reported the existence of more than 97.000 mHealth apps across 62 app stores, with the top 10 apps generating up to 4 million free and 300.000 paid downloads per day. According to a report from MarketsAndMarkets~\cite{marketsandmarkets}, this market is expected to grow even further, from a \$6.21 billion revenue in 2013 to \$23.49 billion by 2018. Because Android is the most attractive mobile platform for malware~\cite{fsecure}, and because, according to Reuters~\footnote{http://www.reuters.com/article/2014/09/24/us-cybersecurity-hospitals-idUSKCN0HJ21I20140924}, medical data is more valuable than credit card information, the mobile health sector is becoming an interesting market for attackers.

To prevent negligent development of health care systems, and protect sensitive data from being disclosed, regulatory laws such as the \ac{HIPAA} have been established. These laws comprise the standard for electronic health care transactions and must be followed by all developers when managing sensitive health data. But regulatory laws are not enough to protect sensitive data, and criminals are motivated by the valuable health care information to perform critical attacks on hospital networks around the world, consequently leaking or stealing health records of millions of patients. In September 2014, a group of hackers attacked UCLA's Hospital network, accessing computers with sensitive records of 4.5 million people. According to \ac{CNN}~\footnote{http://money.cnn.com/2015/07/17/technology/ucla-health-hack/}, among the stolen records were the names, medical information, Social Security numbers, Medicare numbers, health plan IDs, birthdays and physical addresses of UCLA's patients.

% ** PREVIOUS SOLUTIONS AND TECHNICAL DIFFICULTIES **
In the mobile context, data is even more exposed and vulnerable due to the inherent portability of these devices, the sharing of information to third-party advertisers by device manufacturers or mobile app developers, unregulated management of sensitive medical information, specially because regulatory laws such as \ac{HIPAA} do not account for the mobile sector, and because of security flaws on medical or consumer device software.
%REVER ESTA PARTE
To protect sensitive data on mobile apps developers rely on mechanisms such as access control mechanisms, application communication monitoring and privacy enhancement systems, either from native Android, iOS and Windows or from extensions. These mechanisms rely on ad-hoc \ac{OS} and application-level methodologies, which in most cases depend upon a very complex \ac{TCB} code, and do not fully enjoy the potential of the hardware of most modern smartphones, by not taking advantage of technology such as ARM's TrustZone.

%Unlike most popular mobile platforms based on Android~\footnote{https://www.android.com/}, iOS~\footnote{http://www.apple.com/ios/} or Windows~\footnote{http://www.microsoft.com/en-us/windows}, which have a \ac{TCB} comprising of a full featured \ac{OS} and system libraries with millions of \ac{LOC}, a small, dedicated runtime is easy to ensure the absence of exploitable code vulnerabilities. The difficulty in developing a runtime comprised of a small, secure \ac{TCB} is the limit it imposes upon the mobile apps supported. By removing complexity one removes functionality such as networking, I/O and compatibility with most used platforms.

\subsection{Goals}

%Restrições\\

%Contribuições\\

% A discutir melhor depois do trabalho relacionado, comentar por agora?
% ** OBJECTIVES ** 
The goal of this project is to fill in the security gap in the mobile application market by proposing \emph{TrubiZone}, a development system with a small, dedicated \ac{TCB} which, by using ARM TrustZone technology, allows app developers to securely display sensitive content in several formats. With \emph{TrubiZone} the user is guaranteed  to access non-compromised data which is isolated from a rich OS environment such as Android.
%
%% RESTRICTIONS
To support the features referenced above, the final implementation must guarantee the following requirements:
%
\paragraph*{\textbf{Assure Security Policies\\}} \emph{TrubiZone} must guarantee the fundamental security properties of confidentiality, integrity and authenticity, on which every high level security application can be built on.

\paragraph*{\textbf{Trusted User Interfaces\\}} To allow the development of realistic mobile applications, secure user interfaces must be supported, while trying to maintain a small, compact \ac{TCB}.

\paragraph*{\textbf{Support for General Applications\\}} The implementation should run general applications instead of dedicated and specifically crafted apps. To support this restriction \emph{TrubiZone} must execute applications built using standard trusted execution environment APIs.

\paragraph*{\textbf{Developer Friendly\\}} Developers must be able to easily specify the security properties required and the security-sensitive app logic to transparently run in the trusted environment. This allows developers to focus on the logic and design of said application rather than mobile security mechanisms.\\\\
%
%% APPLICATION SPECIFICATION
%TrubiZone will be based on Genode~\footnote{http://genode.org/index} running along side Android and implemented upon a Freescale i.MX53 START development board~\footnote{http://www.freescale.com/products/arm-processors/i.mx-applications-processors-based-on-arm-cores/i.mx53-processors/i.mx53-quick-start-board:IMX53QSB}, which supports ARM's TrustZone. To illustrate TrubiZone's functionality a \ac{mHealth} will be implemented.\\
%
%
%% CONTRIBUTIONS **
In summary this work expects to contribute with:
\begin{itemize}
	\item The design of a novel security system, based on TrustZone, for development of secure mobile applications.
	\item Implementation, on a development board, of a prototype of this new security framework.
	\item Implementation of a mobile health application using the prototyped framework.
	\item Assessment of the prototype and mobile health app.
\end{itemize}

%
%% ORGANIZATION **
The remainder of this document proceeds as follows. Section~\ref{sec:relatedWork} highlights the related work on mobile health security, general-purpose mobile security mechanisms and TrustZone-based systems. Section~\ref{sec:architecture} highlights the architecture of \emph{TrubiZone}. Section~\ref{sec:evaluation} highlights the evaluation methodology and implementation followed by Section~\ref{sec:workplan} which describes the work plan for this project. And Section~\ref{sec:conclusion} concludes this work.
