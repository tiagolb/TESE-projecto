%!TEX root = ../report.tex

% 
% Introduction
% 

\section{Introduction (2/3pgs)}

%General description of the problem and its context, current solutions, and road map of the project.
%1. Contexto, importância da privacidade (nomeadamente no âmbito da e-Health)
%2. Motivação, explicação dos problemas/desafio técnicos em tornar as apps seguras (foco em e-Health)
%3. Definir objectivo e requisitos (dependem dos problemas técnicos)
%4. Apresentar a arquitectura da solução

% ** MOTIVATION AND CONTEXT **
According to a Comscore's whitepaper \cite{comscore} from early 2015, with key statistics for the U.S. market, mobile devices, such as smart-phones and tablets, dominate digital media time over the \ac{PC}, with the trend being to continue raising. This shows a future where mobile devices might be the predominant device.

Present in our everyday personal and professional lives, mobile \ac{apps} start to handle privacy and security-sensitive data, notably photos, health and banking information, location and general documentation, thus becoming an attractive target for attacks.

This paper will focus, for the remaining of this project, on the e-Health market, also known as \ac{mHealth}. In 2013, Research2Guidance \cite{research2guidance} reported the existence of more than 97.000 \ac{mHealth} \ac{apps} across 62 app stores, with the top 10 \ac{mHealth} \ac{apps} generating up to 4 million free and 300.000 paid downloads per day. This market is expected to grow even further, from \$6.21 billion in revenue in 2013 to \$23.49 billion by 2018, according to a report from MarketsAndMarkets \cite{marketsandmarkets}.

The health sector is an interesting market for attackers due to its information value. According to Reuters~\footnote{http://www.reuters.com/article/2014/09/24/us-cybersecurity-hospitals-idUSKCN0HJ21I20140924}, ``medical identity theft is often not immediately identified by a patient or their provider, giving criminals years to milk such credentials. That makes medical data more valuable than credit cards''.

Such valuable information as lead to critical attacks on hospital networks around the world, consequently leaking or stealing health records of millions of patients. In September 2014, a group of hackers attacked the network of the \ac{UCLA}'s Hospital accessing computers with sensitive records of 4.5 million people. According to \ac{CNN}~\footnote{http://money.cnn.com/2015/07/17/technology/ucla-health-hack/}, among the stolen records were the names, medical information, Social Security numbers, Medicare numbers, health plan IDs, birthdays and physical addresses of \ac{UCLA}'s patients.

% ** PREVIOUS SOLUTIONS **
In the mobile context, data is even more exposed and vulnerable because of the portability of such devices, unregulated management of sensitive medical information, the sharing of information to third-party advertisers by device manufacturers and mobile app developers and security flaws on medical or consumer device software.

To protect sensitive data, \ac{apps} rely on ad-hoc \ac{OS} and application-level methodologies, which in most cases depend upon a very complex \ac{TCB} code. Unlike most popular mobile platforms based on Android~\footnote{https://www.android.com/}, iOS~\footnote{http://www.apple.com/ios/} or Windows~\footnote{http://www.microsoft.com/en-us/windows}, which have a \ac{TCB} comprising of a full featured \ac{OS} and system libraries with millions of \ac{LOC}, in a small, dedicated runtime, consisting of only the necessary code to run the intended \ac{apps}, is easy to ensure the absence of exploitable code vulnerabilities.

% OBJECTIVES **
This project aims to fill the security gap in the mobile application market by proposing TrubiZone, a development system with a small, dedicated \ac{TCB} which, by using ARM TrustZone technology, allows app developers to execute parts of the application logic in a trusted environment isolated from the \ac{OS}, thus supporting the development of secure mobile applications independent from the full blown platforms and its inherent vulnerabilities.

TrubiZone will be based on Genode~\footnote{http://genode.org/index} and implemented upon a Freescale i.MX53 START development board, which supports ARM's TrustZone.

% RESTRICTIONS

% APPLICATION SPECIFICATION
To illustrate TrubiZone's functionality an \ac{mHealth} will be implemented.

% ARCHITECTURE

Two worlds etc...

% CONTRIBUTIONS **
In summary this work expects to contribute with:
\begin{itemize}
	\item The design of a novel security system, based on TrustZone, for development of secure mobile applications.
	\item Implementation, on a development board, of a prototype of this new security framework.
	\item Implementation of a mobile health application using the prototyped framework.
	\item Assessment of the prototype and \ac{mHealth} app.
\end{itemize}

% ORGANIZATION **
The remainder of this document proceeds as follows. Section 2 highlights the related work on mobile security, mobile health and TrustZone technology.
