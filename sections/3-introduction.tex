%!TEX root = ../report.tex

% 
% Introduction
% 
% (2/3pgs)
\section{Introduction}

% ** MOTIVATION AND CONTEXT **

Mobile devices are becoming the predominant platform for simple everyday computing activities.  A recent study~\cite{comscore} from early 2015, with key statistics for the U.S. market, shows that smartphones and tablets dominate digital media time over the \ac{PC}, i.e., Internet searches, games and other digital content consumption, with the trend being to continue raising. This shift is explained by the fact that actions that required powerful desktop computers can now be easily performed on mobile devices.

Along with the proliferation of mobile devices, mobile applications (apps) have started to handle privacy and security-sensitive data, such as photos, health records, banking information, location data and general documentation. Mobile health (mHealth) applications, for example, handle medical data, which, according to Reuters~\cite{reuters}, is more valuable than credit card information. In 2013, Research2Guidance~\cite{research2guidance} reported the existence of more than 97.000 mHealth apps across 62 app stores, with the top 10 apps generating up to 4 million free and 300.000 paid downloads per day. This market is expected to grow even further, from a \$6.21 billion revenue in 2013 to \$23.49 billion by 2018~\cite{marketsandmarkets}.

However, as mobile devices store and process increasing amounts of security sensitive data, specially from banking and health monitoring applications, they become more attractive targets for data stealing malware. Data becomes particularly vulnerable when encryption cannot be used, namely when it must be displayed on the device's screen. For example, when browsing through sensitive health records, the data must be written in the framebuffer in its unencrypted raw form so that it is readable by the user. Thus, in a malware-infected operating system, data can easily fall prey to an attacker who can steal it from the framebuffer and leak it out.

Currently, providing secure output capability for commodity mobile platforms, such as Android or iOS, is very challenging. Essentially, to protect sensitive data, mobile app developers rely on mechanisms implemented by the operating system. Such mechanisms include access control~\cite{smalley2013security,kern2012permission,conti2011crepe,heuser2014asm}, application communication monitoring~\cite{ongtang2012semantically,dietz2011quire,bugiel2011xmandroid} and privacy enhancement systems~\cite{beresford2011mockdroid,zhou2011taming,shebaro2014identidroid,enck2014taintdroid}, either from native Android, iOS and Windows or from extensions. But these mechanisms do not offer support for trusted user interfaces, i.e., to securely input and display sensitive information, and rely on ad-hoc \ac{OS} and application-level methodologies, which in most cases depend upon a very complex \ac{TCB} code.

To reduce the \ac{TCB} and offer an isolated environment for security critical applications, the research community has developed trusted hardware technologies~\cite{trustzone_whitepaper}, trusted execution environments~\cite{genode,knox_whitepaper,mcgillion2015open,sierra_tee}, based on microkernels, and trusted services~\cite{li2014droidvault,brasserregulating,li2014building,sun2015trustotp}. But these systems generally do not support, or make it very hard for application developers to offer secure display seamlessly integrated into the application's user interface.

\subsection{Goals}
% ** OBJECTIVES ** 
The goal of this project is to fill in a security gap in the mobile application market by proposing \emph{ViewZone}, a software system that provides trusted display capability for Android applications. Such a capability will enable applications to download encrypted pieces of content, such as images and text, and render them on screen without the need to trust the code of both the application and the operating system. Decrypting and rendering the content will be performed by the \emph{ViewZone} software living in an execution environment isolated from the operating system. To enforce isolation, \emph{ViewZone} will leverage the security primitives of ARM TrustZone, a trusted hardware technology widely available on commodity mobile devices. This ensures that the data displayed by the device is not intercepted by a malware-infected operating system. Thus, the central goal of this thesis is the design and implementation of \emph{ViewZone}.

While providing a trusted display service, \emph{ViewZone} must fulfil the following additional requirements:

\paragraph*{\textbf{Small Trusted Computing Base\\}} To ensure the security of \emph{ViewZone} and of the trusted services running on top of it, this implementation must have a small \ac{TCB}, comprising of a few KLOCs. This can be done by using TrustZone to isolate the secure service responsible for the trusted display from the rich operating system and by designing a carefully crafted trusted kernel responsible for rendering the protected content on screen.

\paragraph*{\textbf{Support for General Applications\\}} The implementation should allow generic Android applications to securely display sensitive content without relying on operating system mechanisms, which may be compromised. Additionally, the system should support the coexistence of trusted and untrusted output, thus allowing the user not to be aware of a context change between the untrusted generic application and its trusted counterpart.

\paragraph*{\textbf{Developer Friendly\\}} Developers must be able to easily specify what needs to be displayed in the trusted environment. This can be done by supporting simple and familiar programming abstractions to render protected content in their apps. This way the system can be used as a secure display service for Android applications.\\

%% CONTRIBUTIONS **
In summary this work expects to contribute with:
\begin{itemize}
	\item[$\bullet$] The design of a TrustZone-based software system that provides secure output channels for Android applications while depending on a small TCB;
	\item[$\bullet$] Implementation, on a development board, of the framework prototype;
	\item[$\bullet$] Implementation of a mHealth application using the prototyped framework;
	\item[$\bullet$] Experimental evaluation of the prototype;
	\item[$\bullet$] Evaluation of tthe mobile health app as a system use case.
\end{itemize}

%
%% ORGANIZATION **
The remainder of this document proceeds as follows. Section~\ref{sec:relatedWork} discusses the related work. Section~\ref{sec:architecture} introduces the architecture of \emph{ViewZone}. Section~\ref{sec:evaluation} highlights the evaluation methodology and implementation. Section~\ref{sec:workplan} describes the work plan for this project. Section~\ref{sec:conclusion} concludes this report.