%!TEX root = ../report.tex

% 
% Abstract 
% 

\begin{abstract}

Mobile devices are growing and being adopted everywhere. Increasingly, such devices execute critical applications which require very sensitive data to be displayed on the devices' screen, for example medical and financial records, which are commonly processed by mHealth and e-money apps, respectively. But, in order to display sensitive information to the user, applications currently depend not only on their own code but also in the code of all underlying layers, which include a full-blown operating system. Given that all these software layers result in a very large and complex computing base, it is difficult to ensure that the sensitive application records displayed on screen have not been intercepted and leaked by a malicious agent that has managed to compromise the operating system. The goal of this thesis is to design and implement a TrustZone-based solution that enables applications to securely output sensitive data without relying on a native rich OS. With ARM's TrustZone, these applications rely on a very small amount of trusted code isolated from the rich OS. This report describes the state of the art and introduces a preliminary architecture of our solution: a software system named ViewZone. To motivate and validate the concept of this solution, a critical mobile health application will be implemented.

\end{abstract}