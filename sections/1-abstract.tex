%!TEX root = ../report.tex

% 
% Abstract 
% 

\begin{abstract}
	
%With the ever-growing number of connected devices worldwide, and with a more conscious and sharing society, mobile devices are becoming interesting data banks. With such an impact in our lives, mobile developers must focus on developing secure and privacy aware applications to protect users and services. Today's mobile operating systems do not offer fully secure methods to support critical applications, such as e-Health, e-voting and e-money, by not taking advantage of secure hardware technology like TrustZone. This paper proposes a new model for the development of critical mobile applications and a system based on TrustZone implementing it.

%Como esse código é muito complexo, é muito difícil garantir a sua correcção e portanto offerecer garantias elevadas de segurança. Neste trabalho o objectivo é conceber e implementar uma solução que permita a aplicações genéricas fazer output seguro de dados sensíveis sem ter que confiar na aplicação e no sistema operativo nativo, mas dependendo de uma TCB reduzida. A abordagem principal será baseada em ARM TrustZone. Para motivar esta primitiva, vamos implementar uma aplicação do foro do mobile health.

Mobile devices are growing and being adopted everywhere. Because of this mobile device ubiquity, critical applications, which require very sensitive data to be displayed, are proliferating, such as mHealth and e-money apps. But, in order to display sensitive information to the user, applications rely not only in its own code but the code of all the underlying layers, which include a full-blown operating system. These rich operating systems have, generally, a very large and complex computing base, which makes it difficult to assess its correctness and ensure security guarantees to critical mobile applications. This paper proposes and implements a solution for generic applications to securely output sensitive data without the need to rely on a native rich OS, but rather on a reduced TCB built using ARM's secure hardware technology, TrustZone. To motivate and prove the concept of this solution, a critical mobile health application is implemented.

\end{abstract}