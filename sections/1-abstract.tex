%!TEX root = ../report.tex

% 
% Abstract 
% 

\begin{abstract}
	
%With the ever-growing number of connected devices worldwide, and with a more conscious and sharing society, mobile devices are becoming interesting data banks. With such an impact in our lives, mobile developers must focus on developing secure and privacy aware applications to protect users and services. Today's mobile operating systems do not offer fully secure methods to support critical applications, such as e-Health, e-voting and e-money, by not taking advantage of secure hardware technology like TrustZone. This paper proposes a new model for the development of critical mobile applications and a system based on TrustZone implementing it.

%Como esse código é muito complexo, é muito difícil garantir a sua correcção e portanto offerecer garantias elevadas de segurança. Neste trabalho o objectivo é conceber e implementar uma solução que permita a aplicações genéricas fazer output seguro de dados sensíveis sem ter que confiar na aplicação e no sistema operativo nativo, mas dependendo de uma TCB reduzida. A abordagem principal será baseada em ARM TrustZone. Para motivar esta primitiva, vamos implementar uma aplicação do foro do mobile health.

Mobile devices are growing and being adopted everywhere. Increasingly, such devices execute critical applications which require very sensitive data to be displayed on the devices' screen, for example medical and financial records, which are commonly processed by mHealth and e-money apps, respectively. But, in order to display sensitive information to the user, applications currently depend not only on their own code but also in the code of all underlying layers, which include a full-blown operating system. Given that all these software layers result in a very large and complex computing base, it is difficult to ensure that the sensitive application records displayed on screen have not been intercepted and leaked by a malicious agent that has managed to compromise the operating system. The goal of this thesis is to design and implement a solution that enables applications to securely output sensitive data without relying on a native rich OS, but rather depend on a very small amount of trusted software code isolated from the rich OS using ARM's secure hardware technology, TrustZone. This report describes the state of the art and introduces a preliminary architecture of our solution: a software system named ViewZone. To motivate and validate the concept of this solution, a critical mobile health application will be implemented.

\end{abstract}