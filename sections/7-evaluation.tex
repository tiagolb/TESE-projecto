%!TEX root = ../report.tex

% 
% Evaluation
% 
% (1/2pgs)
\section{Implementation and Evaluation}
\label{sec:evaluation}

This section describes the implementation details of the solution as well as how the system is going to be evaluated. \emph{ViewZone} will run along side Android and will be implemented upon a Freescale i.MX53 Quick Start Board development board~\footnote{http://www.freescale.com/products/arm-processors/i.mx-applications-processors-based-on-arm-cores/i.mx53-processors/i.mx53-quick-start-board:IMX53QSB}, which supports ARM's TrustZone. To illustrate \emph{ViewZone}'s functionality a Personal Health Record management mobile application will be implemented. This application allows a user to display and store his personal health information securely, without sensitive information ever being exposed to an untrusted rich OS such as Android.

\subsection{Implementation}

%% ISTO DEVE ESTAR NA ARQUITECTURA?
%% ==== MOBILE HEALTH APP EXEMPLO ====
\paragraph{The \ac{PHR} application} comprises an untrusted client Android application which loads the sensitive encrypted file to a dedicated memory region shared by both worlds. This file is the personal health record to be displayed to the user. The client application then notifies the user the file is loaded and the user may proceed with the reliable world switch by pressing the button dedicated for the \ac{NMI} interrupt. After this world switch, the file in the shared memory region is decrypted and shown to the user. The file can then be stored securely in the untrusted world by saving it and triggering a new reliable world switch, which encrypts the file and stores it in the untrusted rich OS.

\paragraph{Memory Isolation} is possible through a watermarking mechanism available in the i.MX53 QSB. This mechanisms allows  the isolation of secure memory regions from non-secure memory ones designated for the rich OS. This memory isolation is necessary for the implementation of the secure framebuffer and to isolate the decrypted file from the normal world domain after the reliable world switch. The i.MX53 QSB has two banks of RAM, each with 512 MB and the watermarking mechanism can watermark one continuous region of up to 256 MB on each bank, totalling 512 MB of possible secure RAM. This is more than enough to support the framebuffer and most regular sized files.

\paragraph{Reliable Switch} is based on the non-maskable interrupt mechanism, but the i.MX53 QSB does not provide any \ac{NMI} explicitly. For this reason an \ac{NMI} needs to be constructed in order to support reliable world switches. To achieve this the interrupt type of the \ac{NMI} must be assigned as secure in the Interrupt Security Register (TZIC\_INTSEC), which prevents the rich OS to modify the configuration of the \ac{NMI}.
% BD - achei isto já demasiado específico da plataforma, se calhar bastava dizer que é preciso proceder a configurações para que o OS não possa armar barracada
Then, several configuration bits are set to zero so that the rich OS cannot disable, block or intercept the interrupt request made to the ARM processor. Finally the interrupt source, such as a physical button, must be configured as a secure peripheral.

% TALVEZ OLHAR PARA O PAPER DO TRUSTOTP TBM PARA VER COISAS SOBRE AS DRIVERS E FRAMEBUFFERS

\subsection{Evaluation}

%Explain how you are going to show your results (statistical data, cpu performance etc). Answer the following questions:
%\begin{itemize}
%  \item Why is this solution going to be better than others.
%  \item How am I going to defend that it is better.
%\end{itemize}

To evaluate \emph{ViewZone} we shall measure the performance overhead for displaying and storing files, as well as compare the development process for the PHR mHealth app using ViewZone and a similar app built on top of Android. The performance can be measured by using the performance monitor available in the Cortex-A8 processor to count the CPU cycles and then convert the cycles to time by multiplying 1 \emph{ns / cycle}. By conducting each experiment for each of the use cases described several times and averaging the value we can compare this value taken for \emph{ViewZone} with the value measured in the same conditions for a similar applications without using the secure system.

%For the power consumption measurements we may use Monsoon Power Monitor~\footnote{https://www.msoon.com/LabEquipment/PowerMonitor/} and compare the consumption of power when ViewZone is running, when it is not running and when a similar application is running outside ViewZone.


\section{Future Work Plan}
\label{sec:workplan}

Future work is scheduled as follows:

\begin{itemize}
	\item January 9 - March 25: Detailed design and implementation of the
	proposed architecture, including preliminary tests.
	\item March 26 - May 1: Perform the complete experimental evaluation
	of the results.
	\item May 2 - May 10: Write a paper describing the project.
	\item May 11 - June 15: Finish the writing of the dissertation.
	\item June 15: Deliver the MSc dissertation.
\end{itemize}

%\begin{figure}
%	\scalebox{.75}{
%		%			\hspace*{0.6cm}
%		\begin{gantt}[xunitlength=0.7cm,fontsize=\large,titlefontsize=\normalsize,drawledgerline=true]{17}{20}
%			\begin{ganttitle}
%				\titleelement{\bf Weeks}{20}
%			\end{ganttitle}
%			\begin{ganttitle}
%				\numtitle{1}{1}{20}{1}
%			\end{ganttitle}
%			\ganttbar{\large\bf T1}{0}{1} 
%			\ganttbarcon{\large\bf T2}{1}{2}
%			\ganttmilestonecon[color=cyan]{\large\bf M1}{3}
%			
%			\ganttbarcon{\large\bf T3}{3}{8}
%			%				\ganttbarcon{\large\bf Vanilla e-meeting \& e-cinema Development}{6}{5}
%			\ganttmilestonecon[color=cyan]{\large\bf M2}{11}
%			
%			\ganttbarcon{\large\bf T4}{11}{1}
%			\ganttbarcon{\large\bf T5}{12}{2}
%			%				\ganttbarcon{\large\bf e-meeting \& e-cinema Trust Lease Incorporation}{13}{1}
%			\ganttmilestonecon[color=cyan]{\large\bf M3}{14}
%			
%			\ganttbarcon{\large\bf T6}{14}{1}
%			\ganttbarcon{\large\bf T7}{15}{1}
%			\ganttbarcon{\large\bf T8}{16}{1}
%			\ganttmilestonecon[color=red]{\large\bf M4}{17}
%			
%			\ganttbar{\large\bf T9}{0}{19}				
%		\end{gantt}
%	}
%	\caption{Work plan Gantt chart\label{gant}}
%\end{figure}