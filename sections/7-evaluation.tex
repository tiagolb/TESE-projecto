%!TEX root = ../report.tex

% 
% Evaluation
% 
% (1/2pgs)
\section{Implementation and Evaluation}
\label{sec:evaluation}

This section describes the implementation details of the solution as well as how the system is going to be evaluated. \emph{ViewZone} will run along side Android and will be implemented upon a Freescale i.MX53 Quick Start Board development board~\footnote{\url{http://www.freescale.com/products/arm-processors/i.mx-applications-processors-based-on-arm-cores/i.mx53-processors/i.mx53-quick-start-board:IMX53QSB}}, which supports ARM's TrustZone. To fully demonstrate the concept of \emph{ViewZone}, we aim to develop a sensitive mobile health application for displaying Personal Health Records (PHR). This application allows a user to display and store his personal health information securely, without sensitive information ever being exposed to an untrusted rich OS such as Android.

\subsection{Implementation}

\paragraph{The \ac{PHR} application} comprises an untrusted client Android application which loads the sensitive encrypted file to a dedicated memory region shared by both worlds. This file is the personal health record to be displayed to the user. The client application then notifies the user the file is loaded and the user may proceed with the reliable world switch by pressing the button dedicated for the \ac{NMI} interrupt. After this world switch, the file in the shared memory region is decrypted and shown to the user. The file can then be stored securely in the untrusted world by saving it and triggering a new reliable world switch, which encrypts the file and stores it in the untrusted rich OS.

\paragraph{Secure Display} is supported by a self-contained secure display controller in the secure domain and an Image Processing Unit (IPU) driver implemented in the rich OS. When the rich OS is running, the IPU is set as a non-secure device and can transfer data from the non-secure framebuffer to the display device. When the system switches to the secure domain, the secure display controller checks the integrity of the IPU driver, saves its state and resets it as a secure device in order to transfer the data from the secure framebuffer to the display device. Before switching back to the normal world, the controller erases the footprint in the IPU to prevent information leakage, and then restores the device state for the rich OS. This method, which reuses the IPU driver implemented by the rich OS, needs additional code to check the driver's integrity, but maintains a smaller TCB than if a self-contained IPU driver was implemented in the secure world.

\paragraph{Memory Isolation} is possible through a watermarking mechanism available in the i.MX53 QSB. This mechanisms allows the isolation of secure memory regions from non-secure memory ones designated for the rich OS. This memory isolation is necessary for the implementation of the secure framebuffer and to isolate the decrypted file from the normal world domain after the reliable world switch. The i.MX53 QSB has two banks of RAM, each with 512 MB and the watermarking mechanism can watermark one continuous region of up to 256 MB on each bank, totalling 512 MB of possible secure RAM. This is more than enough to support the framebuffer and most regular sized files.

\paragraph{Reliable Switch} is based on the non-maskable interrupt mechanism, but the i.MX53 QSB does not provide any \ac{NMI} explicitly. For this reason an \ac{NMI} needs to be constructed in order to support reliable world switches. To achieve this the interrupt type of the \ac{NMI} must be assigned as secure in the Interrupt Security Register (TZIC\_INTSEC), which prevents the rich OS to modify the configuration of the \ac{NMI}. Then, several configuration bits are set to zero so that the rich OS cannot disable, block or intercept the interrupt request made to the ARM processor. Finally the interrupt source, such as a physical button, must be configured as a secure peripheral.

\subsection{Evaluation}

To evaluate \emph{ViewZone} we will measure the performance overhead for displaying files, as well as compare the development process for the PHR mHealth app using ViewZone and a similar app built on top of Android. The performance can be measured by using the performance monitor available in the Cortex-A8 processor to count the CPU cycles and then convert the cycles to time by multiplying 1 \emph{ns / cycle}. By conducting each experiment for each of the use cases described several times and averaging the value we can compare this value taken for \emph{ViewZone} with the value measured in the same conditions for a similar application without using the secure system. We will also assess the complexity of \emph{ViewZone}'s TCB and the final attack surface.

\section{Future Work Plan}
\label{sec:workplan}

Future work is scheduled as follows:

\begin{itemize}
	\item[$\bullet$] January 9 - March 25: Fully design and implement the proposed architecture, including preliminary tests;
	\item[$\bullet$] March 26 - May 1: Perform the complete experimental evaluation 	of the system;
	\item[$\bullet$] March 26 - May 10: Write a paper describing the project;
	\item[$\bullet$] May 11 - June 15: Write the dissertation;
	\item[$\bullet$] June 15: Deliver the MSc dissertation.
\end{itemize}